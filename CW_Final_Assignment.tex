\documentclass[12pt]{article}

\usepackage[utf8]{inputenc}
\usepackage{geometry}
\geometry{a4paper, margin=1in}
\usepackage{graphicx}
\usepackage{hyperref}
\usepackage{fancyhdr}
\usepackage{listings} % Added for code highlighting
\usepackage{xcolor}

\lstdefinelanguage{yaml}{
    keywords={true, false, null, yes, no},
    comment=[l]{\#},
    morestring=[b]',
    morestring=[b]",
    stringstyle=\color{red},
    keywordstyle=\color{blue}\bfseries,
    identifierstyle=\color{black},
    commentstyle=\color{green}\itshape,
    sensitive=false
}
% Listings settings for YAML
\lstset{
    basicstyle=\ttfamily\small,
    breaklines=true,
    numbers=left,
    numberstyle=\tiny,
    frame=single,
    captionpos=b,
    keywordstyle=\color{blue}\bfseries,
    language=yaml
}

\setlength{\headheight}{15pt}
\pagestyle{fancy}
\fancyhf{}
\rhead{Computer Workshop Course}
\lhead{Final Assignment}
\rfoot{Page \thepage}

\title{
    \vspace{2in}
    \textbf{Final Assignment :}\\
    \textbf{Integration of Tools and Practices}\\
    \large Iran University of Science and Technology\\
    \large Department of Computer Engineering\\
    \vspace{2in}
}

\author{
    \vspace{0.5in}
    Qazal Arabali\\
    Computer Workshop\\
    \vspace{0.5in}
}

\date{January 23, 2025}
\begin{document}

\begin{titlepage}
    \maketitle
    \thispagestyle{empty}
\end{titlepage}

\newpage

\tableofcontents
\newpage

\section{Git and GitHub}
    \subsection{Repository Initialization and Commits}
    
To initialize the repository and make the first commits, the following steps were taken :

\begin{enumerate}
    \item Logged into \textbf{GitHub} and created a new repository named :
    \texttt{CW\_Final\_Assignment}
    
    \item Configured the repository settings :
    \begin{itemize}
        \item Set the repository visibility to \textbf{Public}.
        \item Checked the box: \texttt{Initialize this repository with a README}.
    \end{itemize}

    \item Cloned the repository to the local system using the command :
    \begin{lstlisting}
    git clone https://github.com/username/CW_Final_Assignment
    \end{lstlisting}

    \item Navigated to the cloned repository directory :
    \begin{lstlisting}
    cd CW_Final_Assignment
    \end{lstlisting}

    \item Created a new LaTeX file named \texttt{CW\_Final\_Assignment.tex} with the initial structure.

    \item Staged and committed the new file:
    \begin{lstlisting}
    git add CW_Final_Assignment.tex
    git commit -m "Added initial LaTeX file"
    \end{lstlisting}

    \item Pushed the changes to the GitHub repository :
    \begin{lstlisting}
    git push
    \end{lstlisting}
\end{enumerate}

\subsection{GitHub Actions for LaTeX Compilation}

To streamline the process of compiling the LaTeX document and managing different stages of updates, two distinct GitHub Actions workflows were created :

\begin{itemize}
    \item \textbf{Workflow for Intermediate Versions :}
    This workflow compiles the LaTeX document and uploads the resulting PDF as an \textbf{Artifact}. It is used for minor updates or ongoing progress, ensuring the document is accessible for review without cluttering the Releases section.
    
    \item \textbf{Workflow for Major Releases :}
    This workflow compiles the LaTeX document and publishes the resulting PDF as part of a \textbf{GitHub Release}. It is triggered only for significant updates or finalized versions to provide stable releases for public access.
\end{itemize}

\subsubsection{Workflow for Intermediate Versions (Artifact Only)}

For intermediate updates or minor progress, the LaTeX document is compiled and uploaded as an \textbf{Artifact}. This ensures that ongoing work is accessible for review and testing without cluttering the Releases section.

\begin{enumerate}
    \item \textbf{Trigger :} This workflow runs on every \texttt{push} to the \texttt{main} branch.
    \item \textbf{Outcome :} The compiled PDF is stored in the \textbf{Actions > Artifacts} section for download.
    \item \textbf{Workflow Configuration :}
    \begin{lstlisting}[language=yaml]
        name: Compile LaTeX and Upload Artifact
        
        on:
          push:
            branches:
              - main
        
        jobs:
          build_latex:
            runs-on: ubuntu-latest
            steps:
              - name: Checkout Repository
                uses: actions/checkout@v3
        
              - name: Compile LaTeX Document
                uses: xu-cheng/latex-action@v2
                with:
                  root_file: CW_Final_Assignment.tex
        
              - name: Upload Compiled PDF as Artifact
                uses: actions/upload-artifact@v3
                with:
                  name: CW_Final_Assignment
                  path: ./CW_Final_Assignment.pdf
        \end{lstlisting}
        
\end{enumerate}

\subsubsection{Workflow for Major Releases}

For significant updates or final versions, the LaTeX document is compiled, and the resulting PDF is published as part of a \textbf{GitHub Release}. This approach ensures that only stable and finalized versions are available in the Releases section.

\begin{enumerate}
    \item \textbf{Trigger :} This workflow runs on every \texttt{push} of a Git tag in the format \texttt{x.x.x} (e.g., \texttt{1.0.0}).
    \item \textbf{Outcome :} The compiled PDF is attached to a new Release in the \textbf{Releases} section.
    \item \textbf{Workflow Configuration :}
    \begin{lstlisting}[language=yaml]
        name: Release Compiled PDF
        
        on:
          push:
            tags:
              - '*.*.*'
        
        jobs:
          build_latex:
            permissions: write-all
            runs-on: ubuntu-latest
            steps:
              - name: Checkout Repository
                uses: actions/checkout@v3
        
              - name: Compile LaTeX Document
                uses: xu-cheng/latex-action@v2
                with:
                  root_file: CW_Final_Assignment.tex
        
              - name: Create Release
                id: create_release
                uses: actions/create-release@v1
                env:
                  GITHUB_TOKEN: ${{ secrets.GITHUB_TOKEN }}
                with:
                  tag_name: ${{ github.ref }}
                  release_name: Release ${{ github.ref }}
                  draft: false
                  prerelease: false
        
              - name: Upload Release Asset
                id: upload_release_asset
                uses: actions/upload-release-asset@v1
                env:
                  GITHUB_TOKEN: ${{ secrets.GITHUB_TOKEN }}
                with:
                  upload_url: ${{ steps.create_release.outputs.upload_url }}
                  asset_path: ./CW_Final_Assignment.pdf
                  asset_name: CW_Final_Assignment.pdf
                  asset_content_type: application/pdf
    \end{lstlisting}
        
\end{enumerate}

\section{Exploration Tasks}
    \subsection{Vim Advanced Features}
    Here are three advanced features of Vim that go beyond the topics covered in class :

    \subsubsection*{Registers}

    Registers in Vim allow you to store multiple pieces of text in separate "clipboard" areas and retrieve them as needed. This can enhance productivity when working with multiple text snippets.

    \begin{itemize}
        \item To copy text to a specific register, use :
        \begin{lstlisting}
        "aY  # Copy text to register 'a'.
        \end{lstlisting}
        \item To paste text from a specific register, use :
        \begin{lstlisting}
        "ap  # Paste text from register 'a'.
        \end{lstlisting}
    \end{itemize}

    \subsubsection*{Macros}

    Macros enable you to record and replay a series of commands, making repetitive tasks easier.

    \begin{itemize}
        \item To start recording a macro :
        \begin{lstlisting}
        q{register}  # Start recording to a register (e.g., 'a').
        \end{lstlisting}
        \item To stop recording :
        \begin{lstlisting}
        q
        \end{lstlisting}
        \item To play back the macro :
        \begin{lstlisting}
        @{register}  # Play the macro stored in the register.
        \end{lstlisting}
    \end{itemize}

    \subsubsection*{Persistent Undo}

    Persistent Undo allows you to keep the undo history of a file even after closing and reopening Vim.

    \begin{itemize}
        \item To enable persistent undo, add the following to your `.vimrc` :
        \begin{lstlisting}[language=vim]
        set undofile
        set undodir=~/.vim/undodir
        \end{lstlisting}
        \item Once enabled, you can navigate the undo history using :
        \begin{lstlisting}
        u       # Undo the last change.
        Ctrl-r  # Redo the last undone change.
        \end{lstlisting}
    \end{itemize}

    \subsection{Memory profiling}
        \subsubsection{Memory Leak}
        \subsubsection{Memory profilers}

    \subsection{GNU/Linux Bash Scripting}
        \subsubsection{fzf}
        \subsubsection{Using fzf to find your favorite PDF}
        \subsubsection{Opening the file using Zathura}

\section{Git and FOSS}
    \subsection{README.md}
    \subsection{Issues}
    \subsection{FOSS contribution}

\end{document}
